\documentclass[a4paper,11pt]{article}
\usepackage{a4wide}
\usepackage{amsmath}
\usepackage{amssymb}
\usepackage{url}
\usepackage{graphicx}
%\usepackage{graphics}
\usepackage[utf8]{inputenc}
\usepackage{acronym}
%\usepackage{algorithmic}
%\usepackage{algorithm}

\newacro{GRASP}{Greedy Randomized Adaptive Search Procedure}
\newacro{VNS}{Variable Neighborhood Search}
\newacro{VND}{Variable Neighborhood Descent}

% \lstset{language=c,showstringspaces=true,numbers=left,numbersep=5pt}

\title{ \LARGE \bf Problem Solving and Search in AI }


\author{
\bf Partik Marschik(0625039) \\
\bf Martin Schwengerer (0625209) }

\begin{document}

\maketitle

\section{Main Algorithm}
We decided to use a \ac{GRASP}. For the local search, we first want to implement a \emph{tabu search}, testing different neighborhoods (for more details, see below). In a subsequent step, depending on the results of our approach, we may think about a combination with a \ac{VNS} or a \ac{VND}.

%Reactive Variable Neighborhood Tabu Search

%Improving simulated annealing with variable neighborhood search to solve the resource-constrained scheduling problem

\subsection{Problem Representation}
Solutions will be represented by a matrix $TTP$.
Rows $i$ in $TTP$ represent the round of a game.
Columns $j$ in $TTP$ represent the first team.
The value $g_{i,j}$ of a cell represents the second team.
If $g_{i,j} > 0$ then the team plays a home-game, else the team plays a out-game.
Since our indices for the matrix start at $0$ but we cannot represent $-0$ we will
store the values for $g_{i,j}$ by using the transformation
$(|g_{i,j}| + 1) \cdot \text{sgn}(g_{i,j})$.

\subsection{Construction Phase}
For the construction of our initial solution, we will use a multi-phase approach similar to ~\cite{Ribeiro04heuristicsfor} where in a first step, traveling patterns with virtual teams are created and in a subsequent step, the real teams are assigned to the virtual teams.

\subsection{Local Search}
In our first prototype, we will implement a tabu search as local search.

\section{Neighborhood}
For our heuristic, we will implement and test different neighborhoods we discovered. Some of these are based on neighborhoods used in other heuristics (like \cite{Gaspero07,rvk2008}).
For our final solution, it may be that not all of these neighborhoods are used. Moreover, we would like to evaluate the usability of these different neighborhoods with respect to size, computational effort, feasibility (does it contain only valid solutions for our hard- and soft constraints), the existence of a (simple) delta function, etc.
%It has to be mentioned that we will not use any neighborhood violating our hard constraints or need a complex repair function , like \cite{Gaspero07} used for \emph{swapping the games of a team in different rounds}.

In detail, we will consider the following neighborhoods:
\begin{description}
	\item[swapping home/visitor] swapping the home/visitor state of the two games of two teams
	\item[swapping two teams in a season] (swapping teams in all games) exchanging all games of two teams
	%%%
%	\item[swapping homes and teams] %as in (Di Gaspero and Schaerf, 2007; Van Hentenryck and Vergados, 2006; Biajoli and Lorena, 2006; Ribeiro and Urrutia, 2004; Chen et al., 2007)
	\item[swapping rounds] swapping two complete rounds
	\item[shifting rounds] shifting a complete round to a different day (position)
	\item[partial swap rounds] swapping the games of a team in different rounds, \cite{Gaspero07,HentenryckV06, Chen_anant} needs repair chain
	\item[partial swap games] swapping the games of two team one rounds \cite{Gaspero07, HentenryckV06}, needs repair chain
\end{description}

\section{Problem Relaxing}
In order to expand the search space, we decided to relax some problem constraints by allowing illegal solutions in combination with a penalty system.
Therefore, we divide the problem constraints into \emph{hard} and \emph{soft} constraints. The hard constraints are that each team plays twice against each other team, once at home and once as visitor and that the result is a round-robin tour. The soft constraint deals with the home-away-patterns, as each team has limitations about the number of consequent home (or away)-games.

We will test a constant penalty as well as a dynamic penalty (called \emph{shifting penalty}, changing the weight according to the frequencies of feasible and infeasible configurations. Depending if the previous iterations result in a (not) allowed schedule, the penalty decreases (increases). This dynamic penalty mechanism is used in several other, very successful heuristics, like those by \cite{Anagnostopoulos06} or \cite{Gaspero07}.

\small
\bibliographystyle{alpha}
\bibliography{references}

\end{document}

